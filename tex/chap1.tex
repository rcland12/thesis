\chapter{INTRODUCTION}

\hspace{24pt} The goal of statistics is to use data from a sample to learn as much as possible about a population. However, it is often the case that a sample might be contaminated with individuals not belonging to the target population. Statisticians are always trying to improve sampling methods, sample analysis, and parameter estimation, because the end goal is the most representative sample possible. However, statisticians recognize that these methods cannot be perfected and other methods have to be developed to achieve that goal. This thesis paper hopes to achieve a method for characterizing bias, and behavior of bias, whenever contaminated data is present in a model. We will later define this as a mixture model. Specifically, we will analyze two non-parametric measures of association between two variables in a sample (and populations as we will see). These two measures are Spearman's rank correlation coefficient (Rho) and Kendall's rank correlation coefficient (Tau). There is a long history of these coefficients, and there are a few ways to define them as well. We will also see a general correlation coefficient where Rho and Tau are special cases. Copulas are also introduced in Chapter \ref{chap:background}, where we define equivalent forms of both Rho and Tau in terms of copulas. There is a classification of integrals introduced, called Riemann-Stieltjes integrals, which have integral differentials that are not the identity function. We will use a clever technique to solve these with minimal effort.

The main result in this thesis will be defining and characterizing the bias for both Rho and Tau under mixture models. We will introduce a bivariate mixture model ($\vec{M}$) that consists of bivariate measurements from a valid population ($\vec{V}$) and a contaminating population ($\vec{C}$). A key question we will investigate is how this bias is affected as the contamination changes. The bias will be defined in terms of a mixing proportion, $p$ and will have the form $$\text{Bias}_{\theta}\left(p\right)=\theta_{\vec{M}}-\theta_{\vec{V}}$$ for some statistic $\theta$. A natural question one may ask is what form these equations will take. Will they have a simple closed form? If so, can we classify the direction the bias might take as contamination is introduced? These are both questions that we will explore.

We will apply our results to the bivariate Marshall-Olkin distribution. This distribution has a neat and easy analytical form that we can later define explicitly. Furthermore, the distribution has closed form cumulative distribution functions (CDF), probability density functions (PDF), and copulas. This adds to the simplicity of the analytical solutions. While some calculations are very drawn out and tedious, it is primarily elementary algebra and calculus. To wrap things up, we will demonstrate the technique by visualizing the bias for randomly simulated parameter values.