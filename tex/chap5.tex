\chapter{CONCLUSION}
\hspace{24pt} In conclusion, we have answered the result we set out to show. The bias for both rank correlation methods under mixture models are described as polynomials of degrees two and three. Using that result we elaborated and extended the potential by showing the different cases for both quadratics and cubic. The importance of this comes with application. If a researcher has potentially contaminated data, based on parameters they estimate, they may be able to predict whether the bias is positive or negative relative to the mixing proportion. That is, if their data follows a Marshall-Olkin distribution. However, after laying this foundation for bias analysis of rank correlation under mixtures, it could now be extended using different distributions or even multivariate distributions. A multivariate extension of Rho can be found in Trevor Camper's thesis {\it Essays on Mixture Models}, although his motivation is quite different \cite{camper2019}.

Along with our main results in Theorem \ref{theorem:main}, we also introduced a variety of different techniques throughout that readers may be unaware of. These include copulas, which are used extensively in the financial world, a generalized correlation coefficient, population extensions for Rho and Tau, Li's Theorem for CDFs, and some applications of Riemann-Stieljes integrals. The motivation of this project was to explore more techniques that statisticians can add to their toolbox when solving real-world problems. I believe this was accomplished in my thesis report. In reality, techniques are only as good as their ease to use and, with the R code that I provided and the analytical solution to the problems, it could make a successful statistical tool.